\section{Nebenfach} 

\subsection{Study of Famous Figures in Contemporary and Modern China}
This course introduces some of the famous figures in the contemporary and modern China, for example Hu Shi(胡适).

\subsection{Electronic Controlling Technology for Automobile}
This course introduces structure of the control system in a automobile, including chassis control system and motor control system. And some technology applied to the chassis control, such as ABS, ESP are introduced.

\subsection{State-of-the-art of pattern}
This course introduces several algorithms in pattern recognition and its math in detail, including support vector machine(SVM), artificial neural network(ANN), Perceptron, Principal Component Analysis(PCA), Clustering, and Hidden Markov Model(HMM).

\subsection{Drive and Control of Integration Motor System}
This course gives the basic of electric motor, and technical introduction of vector control. Basically, a motor is composed of rotor, stator, air gap, windings, and motor control system.

\subsection{Infrared spectroscopy measuring instrument design}
This is a innovation course, mainly focus on experiment. By the end of the course, I have learnt the concept of infrared spectrum, how to calibrate spectroscopy. The sampled data are fitted with the true value via linear least square method.

\subsection{C Program of Control and Measurement System}
This course talks about system programming in control system. Devices includes: AD/DA and Step motor. The operating system is DOS, IDE is TuroboC. Experiments accounts for half the course, including DOS system programming and timer control, ADC and plotting, wave generation and DAC, step motor control.

\subsection{German as Second Foreign Language I}
This course introduces the very basis of German. In this course, I learnt about the spelling of German and its sentence structure, noun.

\subsection{German as Second Foreign Language II}
This course introduces Adjektiv, Dativ and Akkusativ, and in the end, the teacher introduced some basic concept of Genitiv.

\subsection{Piano Recital art entry}
This is a elective course. The teacher guided me how to play piano and left the most of time to practise. By the end of the course, I managed to play with both hands.

\subsection{An Introduction to Artificial Intelligence}
This course is about natural language processing. It introduces the N Gram module, Viterbi dynamic programming algorithm and Chinese word segmentation. By the end of the course, I managed to make a programming to perform Chinese word segmentation.

\subsection{Algorithm for Big Data}
This course introduces the concept of big data and its algorithms. The characteristics are volume, velocity, variety and value. Due to the limitation of memory, time, I/O and performance, algorithms are adapted to the enormous amount of data. One example is sub-linear time complexity algorithm. It features that only a small amount of data are stored in the memory, and data are feed into the programme(streaming).

\subsection{Introduction to photonics and optical sensor}
This course talks about Fibre Optical System Design. First the teacher talked about the steps for system design, and then the steps were applied to the fibre one. There are four steps involved:
\begin{enumerate}
  \item Evaluate requirements
  \item Make components selection
  \item Calculate power budget
  \item Calculate rise time budget
\end{enumerate}

\subsection{C/C++ senior advanced cases}
This is a experimental course, and the students group up to make a project from a list given by the teacher. I myself form a group. And by the end of the course, I managed to make a dictionary that supports simple regular expression(front, rear and wildcard).

\subsection{The Art of Piano Performance and Appreciation of Classical Music}

\subsection{Innovation Design of Communication \& Electronic systems} 