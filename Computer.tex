\section{Computer}
\subsection{C Programming Language I}
This is a basic course about computer programming, the teacher lectures with C language, the most widely used in the area of engineering.

\subsection{Principles and Applications of MCU}
This course introduces the use of 8051 MCU. First assembly language is introduced, then the concept of timer, interrupt vector table, GPIO, and then the peripheral devices(LCD, keypad, etc...) A large part of the course is experiment, we make design and implementation on the physical chips, for example matrix keyboard.

\subsection{Project Design in Principles and Applications of MCU}
This is a practical course that requires the student team up and write a programme to control a robotic toy car. One of the three function must be implemented: tracing, speed measurement, and collision avoidance.

I've chosen the tracing one. We've given two linear CCD. On the playground there was a circle marked by black line. CCD returns high voltage when it senses light, and low voltage when dark, which indicates the deflection from the trace.

\subsection{Fundamentals of Engineering Software}
This course introduces the basis of computer software, including:
\begin{itemize}
  \item Network
  \item Operating System
  \item Data Structure
  \item Data Base
  \item Basis of Software Engineering
\end{itemize}

\paragraph{Network}
网络拓扑结构;ISO模型七层体系结构, 每层的功能;ISO模型与TCP/IP模型对应关系及差异;

\paragraph{Operating System}
固定分区分配;可变分区分配;分页、分段和段页存储管理;进程管理过程;设备管理的原理;

\paragraph{Data Base}
数据库相关概念;E-R图;范式;基本 SQL语言(查询);